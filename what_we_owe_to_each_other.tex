\documentclass{article}
\usepackage[a4paper, total={7in, 10in}]{geometry}

\usepackage{lipsum}

\usepackage{lineno}

\usepackage{arabtex}
\usepackage{utf8}
\setcode{utf8}

% \usepackage{float}
% \usepackage[]{graphicx}
% \graphicspath{{Images/}}
% \usepackage{hyperref}
% \usepackage{subfiles}

% \usepackage{biblatex}
% \bibliography{ref.bib}
% \addbibresource{ref.bib}

% \usepackage{booktabs}
% \usepackage{multicol}

% \usepackage{tabularx}

\setlength\parindent{0pt}

% remove section numbering
% \setcounter{secnumdepth}{0}

\begin{document}
\linenumbers

\pagenumbering{arabic}


\title{What we Owe to Each Other}
\author{Members of the Imam Mahdi Youth}
\date{\today}
\maketitle

\begin{abstract}
      We owe to each other these values that we declare here, to be keepers of our brothers and sisters, as members of the IMY, and as Muslims.
\end{abstract}

\section{Judgement}
We shall not judge anyone who interacts with us.
Judgement is having or displaying an excessive, critical point of view.
It is any form of negative action, or thought, that imposes a negative, or an inferior look on anyone based on their ethnicity, age, education, language, colour, gender, sexuality, appearance, religion, or lack there-of, practice, acts, or thoughts.

While our community acknowledges the objective truth of Islam in all aspects of life, we are in no moral stance to judge anyone based on what we may think to be immoral.

Judgements can be very subtle, and we shall aim to eradicate within us all of them.
An example of this, is upon hearing that a friend watches a show, we say how can you pray and still watch that show.

There is a proper way of guiding someone to the Islamic path. It takes wit and sensitivity, so as avoid any moral superiority. We shall not aim to prove our point; we aim to guide with them.


\section{Backbiting, Slander, Accusation, or \<غیبت>}

Backbiting is revealing something true but private about our brother's lives, and accusation is revealing something that might not even be true.
We shall not reveal any negative fact, whether true or not, of our brothers and sisters.

We shall only reveal good things about one another to protect the dignity of each other.

An example of backbiting is revealing that a person has had a relationship with another.
We have no right to gossip negative and incriminating facts about one another.

\section{Compassion and Love}
We shall fight the urge to treat this group as a friend group.
We shall practise compassionate welcoming towards everyone who may approach us.

Anyone who may walk into our door is a fresh Muslim, and we shall assume no wrong deed against them.
An example of compassion is inviting our friends, strangers, and anyone of nay background to our mosque.
It is exemplified through smiling, feeding, and showing everyone a special \<محبت> (love).

\section{Respectful Inter-gender Relationship}
While we acknowledge that most of us do come to events to meet and socialise, we must keep the dignity, boundaries, and \<حرمت> of our relationships.

We shall respect all our members and especially our female and younger members.
This means giving everyone a \textbf{safe, inclusive} environment free of \textbf{harassment and misconduct}.
\textbf{The IMY responsible members have a duty and are committed to ensuring that each interaction is safe and non-harmful. }
If anyone feels any unsafe behaviour, harm, or harassment, they or anyone else on their behalf, shall let the IMY responsible members know immediately.

We shall have only respectful conversations with one another; this means setting our boundaries clearly and honestly.
More importantly it means to respect other people's boundaries and comfort levels.

An example of a disrespectful interaction, is unwelcome comments about someone's gender or sexuality.
Another example may include unwelcome, and especially disrespectful, advances, especially repeated advances, towards the opposite gender.

An example of respectful interactions are respectful conversations about welcome, and appropriate topics.
% Another example may include respectfully approaching another \underline{adult}, and willingly accepting the response of the other person.

% We shall speak with one another, respectfully, and with pure intention only.

\section{Dignity}
We shall keep a special duty to the sanctity, justice, and dignity of human lives.
This means condemning extremism, \<ظلم>.
Examples of this practice includes condemning and abstaining from any form of anti-semitism, justifying genocides, and unjust governments who commit crimes such as Apartheid and National Socialism.

We shall stand by our teachings of justice, especially in light of our Imams Ali and Hossein, may peace be upon them.
We especially call for the freedom of Iranians against religious oppression, imprisonment, and wrongful punishments.

We shall also keep a special duty to the dignity of mosques.
This means avoiding profane language, profane music, screaming, shouting, and unjust violence.
It also means that we avoid committing acts that may be perceived as removing the dignity of our mosque.
An example of this is playing cards at the mosque.

\section{Reconciliation}
We have generations of members who have been driven away by our hurtful actions.
We shall not assume that our peers have not been hurt, instead we shall embrace any opportunity to ask their forgiveness.
Forgiveness shall be common and readily given asked for by one another.
Our only path to reconciliation is through seeking forgiveness.

\textbf{We shall regularly review our actions, what we have said, what may have been perceived, and to call anyone for forgiveness. }


% \nocite{*}
% \printbibliography

\end{document}
